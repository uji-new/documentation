\documentclass[../ei103948-project-documentation.tex]{subfiles}
\begin{document}

\section{Introducción}

    En este proyecto conjunto con las asignaturas EI1039\footnote{\textbf{EI1039}: Diseño de software} y EI1048\footnote{\textbf{EI1048}: Paradigmas de software}, vamos de desarrollar desde cero un aplicativo web para consultar ítems relativos a una ubicación dada: tiempo, noticias y eventos.\\

    \subsection{Objetivos}
    Por parte del profesorado de las asignaturas antes mencionadas, se propuso la realización de un proyecto conjunto, que abarcara tanto los conocimientos teóricos, como prácticos de ambas asignaturas a las que hacen referencia.\\

    Este proyecto y las asignaturas a las que atañe, pese a ser '\textit{Cada una de las caras de una moneda}' tiene ciertas particularidades cuando nos enfocamos en cada una de las tareas a desarrollar. Es por ello, por lo que al margen de ciertos requisitos funcionales (que posteriormente expondremos), el desempeño del proyecto por asignturas recaerá en requisitos formativos comprendidos tanto en el marco teórico como en el práctico.\\

        \subsubsection{Objetivos - Ingeniería del Software}
        Para la asignatura de Ingeniería del Software los objetivos principales, son los tres siguientes:
            \begin{itemize}
                \item Implementación de al menos dos patrones de diseño, justificando su idoneidad y beneficios.
                \item Implementación de patrón de diseño para las interfaces de usuario (MVC o similar), justificando la usabilidad de la UI resultante.
                \item Justificación del estilo o estructura de la arquitectura de software seleccionada.
            \end{itemize}
        
        \subsubsection{Objetivos - Paradigmas del Software}
        Para la asignatura de Paradigmas del Software los objetivos principales, son los cinco siguientes:
            \begin{itemize}
                \item Desarrollo íntegro del proyecto usando la metodología ATDD.
                \item Definición de interfaces \texttt{Java} (abstracciones) para modelar interacciones entrela aplicación cliente y cada servicio público de información.
                \item Definición de al menos una interfaz \texttt{Java} para garantizar la independencia de la aplicación cliente con respecto a cualquier solución de almacenamiento adoptada para preservar el estado de la sesión.
                \item Definición de al menos una interfaz \texttt{Java} para garantizar la independencia de la aplicación cliente con respecto a los formatos de intercambio de datos y protocolos de comunicación de las API.
                \item Diseño e implementación de una arquitectura desacoplada que garantice la independencia entre la interfaz de usuario, la lógica de negocio, la base de datos y las dependencias externas $\rightarrow$ API.
            \end{itemize}

    La implementación del proyecto tiene una serie de requisitos, tanto básicos como avanzados. Dependiendo de la realización de los mismos, la calificación final será acorde a la magnitud del trabajo. Los requisitos esenciales, son los siguientes:
    
            \begin{enumerate}[I.]
                \item Gestionar un lugar de interés.
                \item Gestionar hasta tres lugares de interés.
                \item Activar servicios API a partir de una lista.
                \item Recuperar el último estado de la aplicación.
            \end{enumerate}

    No obstante, se abordarán estos requisitos (y adicionalmente los avanzados) en el punto siguiente: $\rightarrow$ \underline{\textbf{2. Requisitos del proyecto}}.


    \subsection{Acceso al código y material adicional}

    Para la corrección de este proyecto, el profesorado pide de forma explícita el identificador del último \texttt{commit} de cada uno de los repositorios, para evitar modificaciones o cambios de versión entre la memoria y sus implementaciones.\\

    Los enlaces a los repositorios (privados) están disponibles en formato \texttt{QR} al inicio de este documento. A continuación se detallan los links de acceso a los repositiorios de este proyecto y de forma adicional, el último \texttt{commit} (y la fecha \faIcon{calendar-check} de realización del mismo).
        \begin{enumerate}[\faIcon{github}]
            \item \textbf{\underline{Documentación}}:
                \begin{itemize}
                    \item [\faIcon{code}] : \href{https://github.com/proyecto-new/documentation}{\texttt{https://github.com/proyecto-new/documentation}}
                    \item [\faIcon{calendar-check}] :  \texttt{???????} $\rightarrow$ 7 ene 2022 ??:?? CET
                \end{itemize}
            \item \textbf{\underline{Frontend}}: 
                \begin{itemize}
                \item [\faIcon{code}] : \href{https://github.com/proyecto-new/webapp}{\texttt{https://github.com/proyecto-new/webapp}}
                \item [\faIcon{calendar-check}] :  \texttt{98260d1} $\rightarrow$ 7 ene 2022 17:55 CET
                \end{itemize}
            \item \textbf{\underline{Backend}}: 
                \begin{itemize}
                    \item [\faIcon{code}] : \href{https://github.com/proyecto-new/app}{\texttt{https://github.com/proyecto-new/app}}
                    \item [\faIcon{calendar-check}] : \texttt{009bb85} $\rightarrow$ 7 ene 2022 17:55 CET
                \end{itemize}
            \item \textbf{\underline{Spike}}:
                \begin{itemize}
                    \item [\faIcon{code}] : \href{https://github.com/proyecto-new/spike}{\texttt{https://github.com/proyecto-new/spike}}
                    \item [\faIcon{calendar-check}] : \texttt{ef6d98a} $\rightarrow$ 7 ene 2022 17:56 CET
                \end{itemize}
            
        \end{enumerate}
\end{document}


