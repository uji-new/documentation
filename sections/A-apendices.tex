\documentclass[../ei103948-project-documentation.tex]{subfiles}
\begin{document}
\appendix

\newpage
\section{Referencias}	
\printbibliography[heading=none]

\newpage
\section{Documentación}
A continuación se muestra la documentación consultada para llevar a cabo este proyecto. A diferencia de las Referencias (punto A de este Apéndice), esta documentación se ha consultado en la parte práctica y no es una referencia a un bloque de la documentación, como sí lo son las del anterior punto. Las fuentes (\textit{docs} \faIcon{book}) son\footnote{Al tratarse de fuentes que se actualizan casi mensualmente, las últimas versiones hacen referencia este año 2021 (salvo excepciones). No obstante, hemos incluido la fecha para cada una de ellas tal y como es recomenado en el formato APA de citación.}:

    \subsection{Frontend - \textit{Javascript}}
        \begin{enumerate}
            \item \underline{Base}
                \begin{itemize}
                    \item [\faIcon{book}] \textbf{Node}: N. (2021). \textit{Documentación}. Node.js. \texttt{https://nodejs.org/es/docs/}
                    \item [\faIcon{book}] \textbf{NPM}: \textit{npm Docs}. (2021). Npm Docs. \texttt{https://docs.npmjs.com}
                    \item [\faIcon{book}] \textbf{React}: \textit{Empezando}. (2021). React. \texttt{https://es.reactjs.org/docs/getting-started.html}
                \end{itemize}
            \item \underline{Diseño}
                \begin{itemize}
                    \item [\faIcon{book}] \textbf{Bootstrap}: Otto, M. J. T. (2021). \textit{Introduction}. Bootstrap - Get Started. \texttt{https://getbootstrap.com/docs/4.1/getting-started/introduction/}
                \end{itemize}
            \item \underline{Utilidades}
                \begin{itemize}
                    \item [\faIcon{book}] \textbf{Babel}: \textit{What is Babel? - Babel}. (2021). Babel JS - Documentation. \texttt{https://babeljs.io/docs/en/}
                \end{itemize}
        \end{enumerate}

    \subsection{Backend - \textit{Java}}
        \begin{enumerate}
            \item \underline{Base}
                \begin{itemize}
                    \item [\faIcon{book}] \textbf{JDK 17}: \textit{JDK 17 Documentation - Home}. (2021, 21 octubre). Oracle Help Center. \texttt{https://docs.oracle.com/en/java/javase/17/}
                    \item [\faIcon{book}] \textbf{Maven}: Redmond, J. V. Z. E. (2021, 20 noviembre).\textit{ Maven – Maven Documentation}. Apache Maven Project. \texttt{https://maven.apache.org/guides/}
                    \item [\faIcon{book}] \textbf{Spring} - Spring Boot: Webb, P. D. S. (2021). \textit{Spring Boot Reference Documentation}. Spring Boot Reference Documentation. \texttt{https://docs.spring.io/spring-boot/docs/current/reference/htmlsingle/}
                \end{itemize}
            \item \underline{Utilidades}
                \begin{itemize}
                    \item [\faIcon{book}] \textbf{Lombok}: \textit{Lombok Stable}. (2021). Lombok Stable Documentation. \texttt{https://projectlombok.org/features/all}
                    \item [\faIcon{book}] \textbf{JCABI Aspects}: Bugayenko, Y. (2017, 7 febrero). \textit{jcabi-aspects – Useful Java AOP Aspects}. JCABI Aspects. \texttt{https://aspects.jcabi.com}
                    \item [\faIcon{book}] \textbf{Jasypt} \textit{Jasypt: Java simplified encryption - Jasypt: Java simplified encryption - General Usage}. (2019, 26 mayo). Jasypt. \texttt{http://www.jasypt.org/general-usage.html}
                \end{itemize}
            \item \underline{Base de datos}
                \begin{itemize}
                    \item [\faIcon{book}] \textbf{H2}: \textit{Quickstart}. (2019, 14 octubre). H2 Database. \texttt{https://www.h2database.com/html/quickstart.html}
                    \item [\faIcon{book}] \textbf{JPA} - Spring JPA: Gierke, O. T. D. (2021). \textit{Spring Data JPA - Reference Documentation}. JPA. \texttt{https://docs.spring.io/spring-data/jpa/docs/current/reference/html/\#reference}
                \end{itemize}
            \item \underline{Peticiones}
                \begin{itemize}
                    \item [\faIcon{book}] \textbf{REST Assured}: R. (2021, 21 mayo). \textit{GettingStarted · rest-assured/rest-assured Wiki}. GitHub - REST Assured. \texttt{https://github.com/rest-assured/rest-assured/wiki/GettingStarted}
                \end{itemize}
            \item \underline{Validación}
            \begin{itemize}
                    \item [\faIcon{book}] \textbf{JUnit 5}: Bechtold, S. S. B. (2021).\textit{ JUnit 5.8.1 User Guide. JUnit 5}. \texttt{https://junit.org/junit5/docs/current/user-guide/}
                    \item [\faIcon{book}] \textbf{Hamcrest}: \textit{Hamcrest 2.2 API.} (2018). Hamcrest. \texttt{http://hamcrest.org/JavaHamcrest/javadoc/2.2/}
                    \item [\faIcon{book}] \textbf{Mockito}: \textit{Mockito - mockito-core 4.1.0 javadoc.} (2021). Mockito. \texttt{https://javadoc.io/doc/org.mockito/mockito-core/latest/org/mockito/Mockito.html}
                \end{itemize}
        \end{enumerate}



\newpage
\section{Glosario}
\subsection{Glosario descriptivo}
        A continuación y lexicográficamente, se muestran breves definiciones a vocablos de interés para la comprensión de esta memoria:
    \begin{itemize}
        \item \textbf{Imaginarium}: Imagen simbólica a partir de la que se desarrolla una representación mental.
        \item \textbf{Isotipo}: Identificador gráfico que por si mismo no requiere ningún texto adicional para definir a la empresa u organización.
        \item \textbf{JUnit}: es un conjunto de bibliotecas utilizadas en programación para hacer pruebas unitarias de aplicaciones Java.
        \item \textbf{Monoespaciada} (Tipografía): Es aquella en la que todos sus caracteres ocupan el mismo espacio horizontal.
        \item \textbf{Punto} (pt.): es una medida utilizada en tipografía y diseño. Cada punto equivale a $\approx$ \texttt{0,0138} pulgadas.
        \item \textbf{React}: es una biblioteca Javascript de código abierto diseñada para crear interfaces de usuario con el objetivo de facilitar el desarrollo de aplicaciones en una sola página.
        \item \textbf{Request} (dicho de una API): Solicitación de un recurso, como un archivo \texttt{.html}, o un \texttt{.json} de un servidor web.
        \item \textbf{Serifa}: o gracia, es un trazo decorativo que forma la terminación de las astas de los caracteres de algunas tipografías. 
        \item \textbf{Spike}: Es un elemento del \textit{Product Backlog} orientado a la investigación o experimentación, cuya finalidad es obtener el aprendizaje necesario para implementar la funcionalidad solicitada.
    \end{itemize}


\subsection{Siglas y acrónimos}
    \begin{itemize}
        \item \textbf{API}: \textbf{A}plication \textbf{P}rogramming \textbf{I}nterfaces
        \item \textbf{ATDD}: \textbf{A}cceptance \textbf{T}est \textbf{D}riven \textbf{D}evelopment
        \item \textbf{GUI}: \textbf{G}raphical \textbf{U}ser \textbf{I}nterface
        \item \textbf{HTTP}: \textbf{H}yper\textbf{T}ext \textbf{T}ransfer \textbf{P}rotocol
        \item \textbf{HU}: \textbf{H}istoria de \textbf{U}suario
        \item \textbf{JDK}: \textbf{J}ava \textbf{D}evelopment \textbf{K}it
        \item \textbf{JPA}: \textbf{J}ava \textbf{P}ersistence \textbf{A}PI
        \item \textbf{JSON}: \textbf{J}ava\textbf{S}cript \textbf{O}bject \textbf{N}otation
        \item \textbf{MVC}: \textbf{M}odelo \textbf{V}ista \textbf{C}ontrolador
        \item \textbf{NPM}: \textbf{N}ode \textbf{P}ackage \textbf{M}anager
        \item \textbf{RA}\textit{X}: \textbf{R}equisito \textbf{A}vanzado \textit{X}
        \item \textbf{RB}\textit{X}: \textbf{R}equisito \textbf{B}ásico \textit{X}
        \item \textbf{REST} (API): \textbf{Re}presentational \textbf{S}tate \textbf{T}ransfer 
    \end{itemize}
\end{document}