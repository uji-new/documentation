\newcommand\descripcionAvanzadaA{\textbf{\underline{Análisis}}\justify
    El escenario E1 solo se podría implementar de una manera, ya que al no haber ambigüedad o estado externo hay poca variación. La única prueba dentro de los parámetros del escenario sería crear una cuenta con un usuario y contraseña válida, y este registrándose.\\ 

    El escenario E3 solo se podría implementar de una manera, ya que al no haber ambigüedad o estado externo hay poca variación. La única prueba dentro de los parámetros del escenario sería crear una cuenta con un usuario ya existente y una contraseña válida, y esté dar error al intentarlo.\\ 
    
    \centering
}
\newcommand\descripcionAvanzadaB{\textbf{\underline{Análisis}}\justify
    El escenario E1 solo se podría implementar de una manera, ya que al no haber ambigüedad o estado externo hay poca variación. La única prueba dentro de los parámetros del escenario sería eliminar una cuenta teniendo una sesión iniciada, y este darse de baja.\\ 

    El escenario E2 solo se podría implementar de una manera, ya que al no haber ambigüedad o estado externo hay poca variación. La única prueba dentro de los parámetros del escenario sería eliminar una cuenta sin tener sesión iniciada, y este dar error al intentarlo.\\ 
    
    \centering

}
\newcommand\descripcionAvanzadaC{\textbf{\underline{Análisis}}\justify
    El escenario E1 solo se podría implementar de una manera, ya que al no haber ambigüedad o estado externo hay poca variación. La única prueba dentro de los parámetros del escenario sería teniendo sesión y una nueva contraseña válida, está cambiarse.\\ 

    El escenario E3 solo se podría implementar de una manera, ya que al no haber ambigüedad o estado externo hay poca variación. La única prueba dentro de los parámetros del escenario sería sin tener sesión ni una contraseña válida, esta no cambiar ningún estado del sistema.\\ 
    
    \centering

}
\newcommand\descripcionAvanzadaD{\textbf{\underline{Análisis}}\justify
    El escenario E1 solo se podría implementar de una manera, ya que al no haber ambigüedad o estado externo hay poca variación. La única prueba dentro de los parámetros del escenario sería iniciar sesión con un usuario y contraseña válida, y esté crear una sesión.\\ 

    El escenario E2 solo se podría implementar de una manera, ya que al no haber ambigüedad o estado externo hay poca variación. La única prueba dentro de los parámetros del escenario sería iniciar sesión con un usuario y contraseña inválidas, y esté dar error al intentarlo.\\ 
    
    \centering

}
\newcommand\descripcionAvanzadaE{\textbf{\underline{Análisis}}\justify
    El escenario E1 solo se podría implementar de una manera, ya que al no haber ambigüedad o estado externo hay poca variación. La única prueba dentro de los parámetros del escenario sería cerrar sesión teniendo una sesión iniciada, y este eliminarse.\\ 

    El escenario E2 solo se podría implementar de una manera, ya que al no haber ambigüedad o estado externo hay poca variación. La única prueba dentro de los parámetros del escenario sería cerrar sesión sin tener sesión iniciada, y esta seguir sin existir.\\ 
    
    \centering

}
\newcommand\descripcionAvanzadaF{%No hay análisis para este test

}

\newcommand\descripcionAvanzadaG{\textbf{\underline{Análisis}}\justify
    El escenario E1 podría subdividirse en al menos tres pruebas de aceptación diferentes. Una con un servicio y ninguna ubicación activa, otra con un servicio y una ubicación activa, y otra con un servicio y múltiples ubicaciones.\\

    El escenario E3 podría subdividirse en al menos tres pruebas de aceptación diferentes. Una con un servicio y ninguna ubicación activa, otra con un servicio y una ubicación activa, y otra con un servicio y múltiples ubicaciones.\\ 
    
    \centering
    
}

\newcommand\descripcionAvanzadaH{\textbf{\underline{Análisis}}\justify
    El escenario E1 podría subdividirse en al menos tres pruebas de aceptación diferentes. Una con un prefijo válido de cuatro caracteres, otra con un prefijo válido de más de cuatro caracteres pero incompleto, y otra con el nombre completo de la ubicación.\\ 

    El escenario E2 solo se podría implementar de una manera, ya que al no haber ambigüedad o estado externo hay poca variación. La única prueba dentro de los parámetros del escenario sería usar una ubicación inválida y esta no responder ninguna sugerencia.\\ 
    
    \centering

}
\newcommand\descripcionBasicaA{\textbf{\underline{Análisis}}\justify
    El escenario E1 podría subdividirse en al menos tres pruebas de aceptación diferentes. Una usando topónimos válidos pero poco conocidos (Antártida), otra con topónimos válidos con una única representación (Madrid) y otra con topónimos válidos con múltiples representaciones (Castelló y Castellón).\\ 

    El escenario E2 podría subdividirse en al menos tres pruebas de aceptación diferentes. Una usando un topónimo invlido formado por símbolos y otros caracteres especiales, otra con un topónimo invlido formado por caracteres normales y otra con un topónimo invalido formado por una pequeña variación de uno valido.\\ 
    
    \centering

}
\newcommand\descripcionBasicaB{\textbf{\underline{Análisis}}\justify
    El escenario E1 podría subdividirse en al menos tres pruebas de aceptación diferentes. Una usando unas coordenadas válidas pero en el límite sintáctico (90.0, 180.0), otra con  unas coordenadas válidas en la frontera de dos ciudades (Castellón y Benicassim) y otra otra  unas coordenadas válidas una ubicación concreta (Castellon).\\ 

    El escenario E2 podría subdividirse en al menos tres pruebas de aceptación diferentes. Una usando unas coordenadas inválidas formadas por símbolos y otros caracteres especiales, otra con unas coordenadas inválidas lejos del límite sintáctico (180.0, 360.0) y otra con unas coordenadas inválidas cerca del límite sintáctico (90.1, 180.1).\\ 
    
    \centering

}
\newcommand\descripcionBasicaC{\textbf{\underline{Análisis}}\justify
    El escenario E1 podría subdividirse en al menos tres pruebas de aceptación diferentes. Una usando topónimos válidos disponibles pero poco conocidos (Antártida), otra con topónimos válidos disponibles conocidos (Castellón), y otra con topónimos válidos disponible que ya esté registrado.\\ 

    El escenario E3 podría subdividirse en al menos tres pruebas de aceptación diferentes. Una usando un topónimo invlido formado por símbolos y otros caracteres especiales, otra con un topónimo invlido formado por caracteres normales y otra con un topónimo invalido formado por una pequeña variación de uno valido.\\ 
    
    \centering

}
\newcommand\descripcionBasicaD{\textbf{\underline{Análisis}}\justify
    El escenario E1 podría subdividirse en al menos tres pruebas de aceptación diferentes. Una usando unas coordenadas válidas pero en el límite sintáctico (90.0, 180.0), otra con  unas coordenadas válidas en la frontera de dos ciudades (Castellón y Benicassim) y otra otra  unas coordenadas válidas una ubicación concreta (Castellon).\\ 

    El escenario E3 podría subdividirse en al menos tres pruebas de aceptación diferentes. Una usando unas coordenadas inválidas formadas por símbolos y otros caracteres especiales, otra con unas coordenadas inválidas lejos del límite sintáctico (180.0, 360.0) y otra con unas coordenadas inválidas cerca del límite sintáctico (90.1, 180.1).\\ 
    
    \centering

}
\newcommand\descripcionBasicaE{\textbf{\underline{Análisis}}\justify
    El escenario E1 solo se podría implementar de una manera, ya que al no haber ambigüedad o estado externo hay poca variación. La única prueba dentro de los parámetros del escenario sería activar una ubicación que ya esté en el sistema pero estuviera desactivado.\\ 

    El escenario E2 solo se podría implementar de una manera, ya que al no haber ambigüedad o estado externo hay poca variación. La única prueba dentro de los parámetros del escenario sería activar una ubicación que ya esté en el sistema pero estuviera activado.\\ 
    
    \centering

}
\newcommand\descripcionBasicaF{\textbf{\underline{Análisis}}\justify
    El escenario E1 podría subdividirse en al menos tres pruebas de aceptación diferentes. Una usando topónimos válidos pero poco conocidos (Antártida), otra con topónimos válidos con una única representación (Madrid) y otra con topónimos válidos con múltiples representaciones (Castelló y Castellón).\\

    El escenario E2 podría subdividirse en al menos tres pruebas de aceptación diferentes. Una usando un topónimo invlido formado por símbolos y otros caracteres especiales, otra con un topónimo invlido formado por caracteres normales y otra con un topónimo invalido formado por una pequeña variación de uno valido.\\ 
    
    \centering

}
\newcommand\descripcionBasicaG{\textbf{\underline{Análisis}}\justify
    El escenario E1 podría subdividirse en al menos tres pruebas de aceptación diferentes. Una usando unas coordenadas válidas pero en el límite sintáctico (90.0, 180.0), otra con unas coordenadas válidas en la frontera de dos ciudades (Castellón y Benicassim) y otra otra  unas coordenadas válidas una ubicación concreta (Castellon).\\

    El escenario E3 podría subdividirse en al menos tres pruebas de aceptación diferentes. Una usando unas coordenadas inválidas formadas por símbolos y otros caracteres especiales, otra con unas coordenadas inválidas lejos del límite sintáctico (180.0, 360.0) y otra con unas coordenadas inválidas cerca del límite sintáctico (90.1, 180.1).\\ 
    
    \centering

}
\newcommand\descripcionBasicaH{\textbf{\underline{Análisis}}\justify
    El escenario E1 podría subdividirse en al menos tres pruebas de aceptación diferentes. Una usando un alias vacío, otra un alias con caracteres en blanco y otra con un alias con caracteres normales.\\ 

    El escenario E2 solo se podría implementar de una manera, ya que al no haber ambigüedad o estado externo hay poca variación. La única prueba dentro de los parámetros del escenario sería no proporcionar un alias y comprobar que se restablece el original.\\ 
    
    \centering

}
\newcommand\descripcionBasicaI{\textbf{\underline{Análisis}}\justify
    El escenario E1 solo se podría implementar de una manera, ya que al no haber ambigüedad o estado externo hay poca variación. La única prueba dentro de los parámetros del escenario sería desactivar una ubicación que ya esté en el sistema pero estuviera activado.\\ 

    El escenario E2 solo se podría implementar de una manera, ya que al no haber ambigüedad o estado externo hay poca variación. La única prueba dentro de los parámetros del escenario sería desactivar una ubicación que ya esté en el sistema pero estuviera desactivado.\\ 
    
    \centering

}
\newcommand\descripcionBasicaJ{\textbf{\underline{Análisis}}\justify
    El escenario E1 solo se podría implementar de una manera, ya que al no haber ambigüedad o estado externo hay poca variación. La única prueba dentro de los parámetros del escenario sería dar de baja una ubicación que ya esté en el sistema.\\ 

    El escenario E2 solo se podría implementar de una manera, ya que al no haber ambigüedad o estado externo hay poca variación. La única prueba dentro de los parámetros del escenario sería dar de baja una ubicación que ya no esté en el sistema.\\ 
    
    \centering

}
\newcommand\descripcionBasicaK{\textbf{\underline{Análisis}}\justify
    El escenario E1 solo se podría implementar de una manera, ya que al no haber ambigüedad o estado externo hay poca variación. La única prueba dentro de los parámetros del escenario sería tener al dos ubicaciones activas dadas de alta y está responder sus datos.\\

    El escenario E2 solo se podría implementar de una manera, ya que al no haber ambigüedad o estado externo hay poca variación. La única prueba dentro de los parámetros del escenario sería no tener ninguna ubicación activa dada de alta y esta no responder.\\

    
    \centering

}
\newcommand\descripcionBasicaL{\textbf{\underline{Análisis}}\justify
El escenario E2 podría subdividirse en al menos dos pruebas de aceptación diferentes. Una naive que mientras exista respuesta de ambos asume que ha funcionado, otra sanity que realice lo mismo pero además comprueba el nombre de los campos y su tipado.\\

El escenario E3 solo se podría implementar de una manera, ya que al no haber ambigüedad o estado externo hay poca variación. La única prueba dentro de los parámetros del escenario sería no tener servicios, ni ubicaciones y este no responder.\\ 
    
    \centering
}
\newcommand\descripcionBasicaM{\textbf{\underline{Análisis}}\justify
El escenario E1 solo se podría implementar de una manera, ya que al no haber ambigüedad o estado externo hay poca variación. La única prueba dentro de los parámetros del escenario sería tener una ubicación guardada sin servicios, un servicio disponible y activar este servicio en la ubicación.\\
 
El escenario E3 solo se podría implementar de una manera, ya que al no haber ambigüedad o estado externo hay poca variación. La única prueba dentro de los parámetros del escenario sería tener una ubicación guardada sin servicios, un servicio disponible y activar un servicio que no sea este en la ubicación.\\

    
    \centering

}
\newcommand\descripcionBasicaN{\textbf{\underline{Análisis}}\justify
El escenario E1 solo se podría implementar de una manera, ya que al no haber ambigüedad o estado externo hay poca variación. La única prueba dentro de los parámetros del escenario sería tener una ubicación guardada con servicios, un servicio disponible y desactivar este servicio en la ubicación.\\
 
El escenario E3 solo se podría implementar de una manera, ya que al no haber ambigüedad o estado externo hay poca variación. La única prueba dentro de los parámetros del escenario sería tener una ubicación guardada con servicios, un servicio disponible y desactivar un servicio que no sea este en la ubicación.\\

    \centering

}
\newcommand\descripcionBasicaO{\textbf{\underline{Análisis}}\justify
El escenario E1 solo se podría implementar de una manera, ya que al no haber ambigüedad o estado externo hay poca variación. La única prueba dentro de los parámetros del escenario sería tener dos ubicaciones guardadas pero solo una activa y al consultar esta devolver solo la activa.\\ 

El escenario E2 solo se podría implementar de una manera, ya que al no haber ambigüedad o estado externo hay poca variación. La única prueba dentro de los parámetros del escenario sería tener dos ubicaciones guardadas pero ninguna activa y al consultar esta no devolver ninguna.\\

    
    \centering

}
\newcommand\descripcionBasicaP{\textbf{\underline{Análisis}}\justify
El escenario E1 solo se podría implementar de una manera, ya que al no haber ambigüedad o estado externo hay poca variación. La única prueba dentro de los parámetros del escenario sería tener dos ubicaciones guardadas y ambas activas, al consultar sobre una concreta esta devolver solo esa.\\

El escenario E2 solo se podría implementar de una manera, ya que al no haber ambigüedad o estado externo hay poca variación. La única prueba dentro de los parámetros del escenario sería tener dos ubicaciones guardadas pero ninguna activa y al consultar cualquiera esta devolver ninguna.\\

    \centering

}
\newcommand\descripcionBasicaQ{\textbf{\underline{Análisis}}\justify
El escenario E1 podría subdividirse en al menos dos pruebas de aceptación diferentes. Una \textit{naive} que mientras exista respuesta de clima asume que ha funcionado, otra \textit{sanity} que realice lo mismo pero además comprueba el nombre de los campos y su tipado.\\

El escenario E2 podría subdividirse en al menos dos pruebas de aceptación diferentes. Una \textit{naive} que mientras exista respuesta de clima asume que ha funcionado, otra \textit{sanity} que realice lo mismo pero además comprueba el nombre de los campos y su tipado.\\
    
    \centering

}
\newcommand\descripcionBasicaR{\textbf{\underline{Análisis}}\justify
El escenario E1 podría subdividirse en al menos dos pruebas de aceptación diferentes. Una \textit{naive} que mientras exista respuesta de eventos asume que ha funcionado, otra \textit{sanity} que realice lo mismo pero además comprueba el nombre de los campos y su tipado.\\

El escenario E2 podría subdividirse en al menos dos pruebas de aceptación diferentes. Una que consulte una ubicación no reconocida y ésta no responda, y otra que realice la prueba desconectado de la red y esta tampoco responda.\\

    \centering

}
\newcommand\descripcionBasicaS{\textbf{\underline{Análisis}}\justify
El escenario E1 podría subdividirse en al menos dos pruebas de aceptación diferentes. Una \textit{naive} que mientras exista respuesta de noticias asume que ha funcionado, otra \textit{sanity} que realice lo mismo pero además comprueba el nombre de los campos y su tipado.\\

El escenario E2 podría subdividirse en al menos dos pruebas de aceptación diferentes. Una que consulte una ubicación no reconocida y ésta no responda, y otra que realice la prueba desconectado de la red y esta tampoco responda.\\

    
    \centering

}
\newcommand\descripcionBasicaT{\textbf{\underline{Análisis}}\justify
El escenario E1 solo se podría implementar de una manera, ya que al no haber ambigüedad o estado externo hay poca variación. La única prueba dentro de los parámetros del escenario sería consultar el histórico y comprobar que hay solo una ubicación.\\

El escenario E2 solo se podría implementar de una manera, ya que al no haber ambigüedad o estado externo hay poca variación. La única prueba dentro de los parámetros del escenario sería consultar el histórico y comprobar que no hay ninguna ubicación.\\

    
    \centering
}
\newcommand\descripcionBasicaU{\textbf{\underline{Análisis}}\justify
    El escenario E1 podría subdividirse en al menos dos pruebas de aceptación diferentes. Una \textit{naive} que mientras exista asume que el servicio está disponible, otra \textit{sanity} que realice una petición de prueba para ver si el servicio funciona como se espera.\\ 

    El escenario E2 podría subdividirse en al menos dos pruebas de aceptación diferentes. Una \textit{naive} que mientras no exista asume que el servicio está no disponible, otra \textit{sanity} que realice una petición de prueba para ver si el servicio funciona mientras está desconectado de la red.\\ 
    
    \centering
}
\newcommand\descripcionBasicaV{\textbf{\underline{Análisis}}\justify
    El escenario E1 solo se podría implementar de una manera, ya que al no haber ambigüedad o estado externo hay poca variación. La única prueba dentro de los parámetros del escenario sería activar un servicio desactivado y este activarse.\\ 

    El escenario E2 solo se podría implementar de una manera, ya que al no haber ambigüedad o estado externo hay poca variación. La única prueba dentro de los parámetros del escenario sería intentar activar un servicio invalido y este no activarse.\\ 
    
    \centering
}
\newcommand\descripcionBasicaW{\textbf{\underline{Análisis}}\justify
    El escenario E1 solo se podría implementar de una manera, ya que al no haber ambigüedad o estado externo hay poca variación. La única prueba dentro de los parámetros del escenario sería consultar un servicio válido disponible y comprobar que devuelve datos.\\ 

    El escenario E2 solo se podría implementar de una manera, ya que al no haber ambigüedad o estado externo hay poca variación. La única prueba dentro de los parámetros del escenario sería consultar un servicio inválido y comprobar que no devuelve datos.\\ 
    
    \centering
}
\newcommand\descripcionBasicaX{\textbf{\underline{Análisis}}\justify
    El escenario E1 solo se podría implementar de una manera, ya que al no haber ambigüedad o estado externo hay poca variación. La única prueba dentro de los parámetros del escenario sería desactivar un servicio activado y este desactivarse.\\ 

    El escenario E2 solo se podría implementar de una manera, ya que al no haber ambigüedad o estado externo hay poca variación. La única prueba dentro de los parámetros del escenario sería intentar desactivar un servicio inválido y este no cambiar el estado.\\ 
    
    \centering
}
\newcommand\descripcionBasicaY{\textbf{\underline{Análisis}}\justify
    El escenario E1 solo se podría implementar de una manera, ya que al no haber ambigüedad o estado externo hay poca variación. La única prueba dentro de los parámetros del escenario sería crear una cuenta y esta crearse modificando la base de datos.\\ 
    
    El escenario E2 podría subdividirse en al menos dos pruebas de aceptación diferentes. Una usando un usuario registrado sin ningún dato adicional, otra usando un usuario registrado con múltiples datos adicionales (como ubicaciones, histórico, servicios…).\\ 
    
    \centering
}